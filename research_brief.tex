\documentclass[11pt]{article}
\usepackage[margin=1in]{geometry}
\usepackage{parskip}
\usepackage{graphicx}
\usepackage{titlesec}
\usepackage{url}
\usepackage{natbib}

\titleformat{\section}{\large\bfseries}{\thesection}{1em}{}
\titlespacing*{\section}{0pt}{1ex}{0.5ex}

\begin{document}

\begin{center}
{\LARGE \textbf{Feng Lab: Clinical AI Innovation and Research for All}}
\end{center}

%\begin{figure}
%	\includegraphics[width=\linewidth]{learning_healthcare_system.png}
%\end{figure}

Artificial intelligence (AI) is now transforming healthcare.
The growing number of AI/ML-based medical devices approved by the FDA is increasing every year, and innovations like AI scribes are now being adopted widely across health systems.
Done thoughtfully, AI can deliver better healthcare to everyone.
Done poorly, AI can actually do more harm and, ultimately, risk bringing forth another AI winter, where the technology is again rejected by society.

The goal of the Feng Lab is to develop AI tools and methods that will help the field overcome this critical juncture.
We do this through a dual approach: half of the lab is dedicated to building AI tools that improve patient care and outcomes, while the other half of the lab researches new AI methods so that these technologies satisfy mathematical guarantees to ensure their safety and effectiveness.
Through this, we ensure that the AI tools we deploy in the hospital are informed by the latest research in AI, and the AI research addresses the most critical problems faced in real-world settings.

\textbf{Building AI Tools.}
Our lab serves as the data science arm of PROSPECT \url{https://zsfg-prospect.github.io/}, the digital innovation taskforce at the Zuckerberg San Francisco General Hospital whose mission is to apply technology and digital tools to improve health outcomes and equity in vulnerable and underserved populations. 
PROSPECT is a highly interdisciplinary team composed of experts in machine learning, artificial intelligence, data science, clinical informatics, hospital organization and management, health equity, and social determinants of health.
Our AI-powered readmission prediction system helped ZSFG reduce their 30-day unplanned readmission rate from one of the highest  among California safety-net hospitals to one of the lowest---saving \$7.2 million in at-risk pay-for-performance funding \citep{Bennett...}.
We have also developed and validated an AI-powered patient dashboard that automatically summarize complex medical charts to identify social needs, which is estimated to increase the efficiency of the social work team by over 50\%.
Working with the hospital leadership team, we are now leveraging AI to help identify the most promising quality improvement initiatives to improve hospital flow, patient outcomes, and workplace satisfaction.

\textbf{Researching new methods.}
Working through UCSF-Stanford Center of Excellence in Regulatory Science and Innovation, our lab closely collaborates with researchers at the US Food and Drug Administration to develop statistical methods that improve the safety and reliability of AI devices, which are used to inform regulatory policies.
Together, our team has written landmark papers on AI monitoring \citep{} and developed methods that are guaranteed to detect performance deterioration of AI models in a timely fashion \citep{}.
We have also developed statistically rigorous methods for benchmarking generative AI tools \citep{}, which are now proliferating throughout health systems.
Most recently, we released open-source AI pipelines for conducting research on regulatory science of AI medical devices at scale to democratize access to regulatory data and, ultimately, improve oversight and safety of these devices \citep{}.

TODO: Some ending paragraph?

%%%%%%%%%%%%%%%%%%%%%%%%%%%%%%%%%%%%%%%%%%%%%%%%%%%%%%%%%%%%%%%
% REVISED VERSION
%%%%%%%%%%%%%%%%%%%%%%%%%%%%%%%%%%%%%%%%%%%%%%%%%%%%%%%%%%%%%%%

\newpage

\begin{center}
{\LARGE \textbf{Feng Lab: Clinical AI Innovation and Research for All}}
\end{center}

%\begin{figure}
%	\includegraphics[width=\linewidth]{.png}
%\end{figure}

Artificial Intelligence (AI) is now transforming healthcare and patient care and, when done thoughtfully, AI can do immense good.
At the Zuckerberg San Francisco General Hospital, our group has helped the hospital become one of the leaders in developing clinical AI tools for vulnerable and underserved populations.
Our AI-powered readmission prediction system helped ZSFG reduce their 30-day unplanned readmission rate from one of the highest among California safety-net hospitals to one of the lowest---saving \$7.2 million in at-risk pay-for-performance funding and, more importantly, keeping patients healthy at home rather than returning to the hospital \citep{Bennett...}.
Our AI dashboard automatically summarizes complex medical charts to surface social-medical needs---housing instability, food insecurity, transportation barriers---and, once deployed, is anticipated to increase the efficiency of the social work team by over 50\% \citep{}.
These successes have led the hospital leadership team to seek the use of AI across wide-ranging tasks, from identifying the most promising initiatives for improving patient outcomes to extracting key factors affecting workplace safety.
% Work from our group has received numerous awards including the Quality Leaders Award in Equity at the California Association of Public Hospitals and Health Systems Safety Net Institute and the Joint Commission’s Bernard Tyson Award for Excellence in Pursuit of Healthcare Equity.

At the same time, it is paramount we also address the major risks posed by AI, including but not limited to worsening health disparities, eroding clinician trust, or silently worsening in its performance.
As such, our lab draws on our real-world AI deployment experience to inform our research on developing new methods that ensure AI algorithms are safe, as characterized by rigorous mathematical guarantees.

Through the UCSF-Stanford Center of Excellence in Regulatory Science and Innovation, our methodological research is done in close collaboration with the US Food and Drug Administration, where together our teams develop statistical methods that inform regulatory policy.
We have become leaders in the area of AI monitoring, including landmark papers on how hospitals can adopt quality improvement tools for AI monitoring \citep{} and novel statistical methods that are guaranteed to detect performance deterioration in a timely fashion \citep{}.
We are also pushing the regulatory science towards developing methods for rigorously assessing generative AI tools \citep{}, which are now proliferating throughout health systems, including those developed by our team.
Furthermore, to democratize access to regulatory data and improving oversight of AI devices, we've developed and released open-source pipelines so that anyone can easily conduct regulatory science research at scale \citep{}.

Looking ahead, our lab seeks to become a true Learning Healthcare System, where real-world challenges inform our research and our research informs our translational efforts.
Our mission is to lead the development and deployment of AI tools at  safety-net hopsitals and the like, so that AI will ultimately benefit everyone.

%%%%%%%%%%%%%%%%%%%%%%%%%%%%%%%%%%%%%%%%%%%%%%%%%%%%%%%%%%%%%%%

\bibliographystyle{unsrt}
\bibliography{research_brief}

\end{document}
