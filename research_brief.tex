\documentclass[11pt]{article}
\usepackage[margin=1in]{geometry}
\usepackage{parskip}
\usepackage{graphicx}
\usepackage{titlesec}
\usepackage{url}
\usepackage{natbib}

\titleformat{\section}{\large\bfseries}{\thesection}{1em}{}
\titlespacing*{\section}{0pt}{1ex}{0.5ex}

\begin{document}

\begin{center}
{\LARGE \textbf{Feng Lab: Clinical AI Innovation and Research for All}}
\end{center}

%\begin{figure}
%	\includegraphics[width=\linewidth]{learning_healthcare_system.png}
%\end{figure}

Artificial intelligence (AI) is now transforming healthcare.
The growing number of AI/ML-based medical devices approved by the FDA is increasing every year, and innovations like AI scribes are now being adopted widely across health systems.
Done thoughtfully, AI can deliver better healthcare to everyone.
Done poorly, AI can actually do more harm and, ultimately, risk bringing forth another AI winter, where the technology is again rejected by society.

The goal of the Feng Lab is to develop AI tools and methods that will help the field overcome this critical juncture.
We do this through a dual approach: half of the lab is dedicated to building AI tools that improve patient care and outcomes, while the other half of the lab researches new AI methods so that these technologies satisfy mathematical guarantees to ensure their safety and effectiveness.
Through this, we ensure that the AI tools we deploy in the hospital are informed by the latest research in AI, and the AI research addresses the most critical problems faced in real-world settings.

\textbf{Building AI Tools.}
Our lab serves as the data science arm of PROSPECT \url{https://zsfg-prospect.github.io/}, the digital innovation taskforce at the Zuckerberg San Francisco General Hospital whose mission is to apply technology and digital tools to improve health outcomes and equity in vulnerable and underserved populations. 
PROSPECT is a highly interdisciplinary team composed of experts in machine learning, artificial intelligence, data science, clinical informatics, hospital organization and management, health equity, and social determinants of health.
Our AI-powered readmission prediction system helped ZSFG reduce their 30-day unplanned readmission rate from one of the highest  among California safety-net hospitals to one of the lowest---saving \$7.2 million in at-risk pay-for-performance funding \citep{Bennett...}.
We have also developed and validated an AI-powered patient dashboard that automatically summarize complex medical charts to identify social needs, which is estimated to increase the efficiency of the social work team by over 50\%.
Working with the hospital leadership team, we are now leveraging AI to help identify the most promising quality improvement initiatives to improve hospital flow, patient outcomes, and workplace satisfaction.

\textbf{Researching new methods.}
Working through UCSF-Stanford Center of Excellence in Regulatory Science and Innovation, our lab closely collaborates with researchers at the US Food and Drug Administration to develop statistical methods that improve the safety and reliability of AI devices, which are used to inform regulatory policies.
Together, our team has written landmark papers on AI monitoring \citep{} and developed methods that are guaranteed to detect performance deterioration of AI models in a timely fashion \citep{}.
We have also developed statistically rigorous methods for benchmarking generative AI tools \citep{}, which are now proliferating throughout health systems.
Most recently, we released open-source AI pipelines for conducting research on regulatory science of AI medical devices at scale to democratize access to regulatory data and, ultimately, improve oversight and safety of these devices \citep{}.

TODO: Some ending paragraph?

%%%%%%%%%%%%%%%%%%%%%%%%%%%%%%%%%%%%%%%%%%%%%%%%%%%%%%%%%%%%%%%
% REVISED VERSION
%%%%%%%%%%%%%%%%%%%%%%%%%%%%%%%%%%%%%%%%%%%%%%%%%%%%%%%%%%%%%%%

\newpage

\begin{center}
{\LARGE \textbf{Feng Lab: Clinical AI Innovation and Research for All}}
\end{center}

%\begin{figure}
%	\includegraphics[width=\linewidth]{.png}
%\end{figure}

Artificial Intelligence (AI) is now transforming healthcare.
At the Zuckerberg San Francisco General Hospital (ZSFG), our group serves as data science arm of the digital innovation team and has shown how AI, when done thoughtfully, can do immense good.
Over the past four years, we have helped ZSFG become a leader in developing clinical AI solutions for vulnerable and underserved populations.
Our AI-based readmission system reduced the hospital's 30-day unplanned readmission rate from one of the highest rates among California safety-net hospitals to one of the lowest---saving \$7.2 million in at-risk pay-for-performance funding and, more importantly, keeping patients healthy at home \citep{Bennett...}.
Our AI-powered dashboard automatically summarizes complex medical charts to surface medical-social needs and is anticipated to increase social work efficiency by over 50\% \citep{}.
These successes have convinced hospital leadership to now leverage AI across an increasing range of quality improvement tasks.
Our work has also received numerous awards, including from the California Association of Public Hospitals and Health Systems Safety Net Institute and the Joint Commission's Award for Excellence in Pursuit of Healthcare Equity.

At the same time, it is paramount we address and remain vigilant regarding the major risks posed by AI, particularly when one is developing AI solutions for vulnerable and underserved populations.
Consequently, our lab is simultaneously dedicated to developing methods that ensure the safety and reliability of AI systems.
Thus, our lab runs two parallel efforts: build AI tools and research AI methods.
These two arms inform each other to embody a Learning Healthcare System, where real-world challenges inform our research and our research drives translational impact.

To ensure our research meets the highest standards of rigor, we collaborate closely with the US Food and Drug Administration to develop new methods and frameworks for evaluating and monitoring AI systems.
Together, our group has become a leader in AI monitoring, publishing landmark papers on quality improvement frameworks for AI oversight \citep{} and novel methods guaranteed to detect performance deterioration in a timely fashion \citep{}.
Given the rapid rise of generative AI tools, we are also now working with the FDA to research regulatory-grade methods for evaluating free-form outputs from these systems \citep{}.
More broadly, in our effort to democratize regulatory science research, we have released open-source pipelines enabling anyone to analyze and extract information from regulatory documents at scale \citep{}.

NEEDS A FINAL PARAGRAPH
% old sentence: Our mission is to lead AI development and deployment at safety-net hospitals, ensuring that AI benefits everyone.

%%%%%%%%%%%%%%%%%%%%%%%%%%%%%%%%%%%%%%%%%%%%%%%%%%%%%%%%%%%%%%%

\bibliographystyle{unsrt}
\bibliography{research_brief}

\end{document}
